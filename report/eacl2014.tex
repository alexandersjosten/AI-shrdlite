%
% File eacl2014.tex
%
% Contact g.bouma@rug.nl yannick.parmentier@univ-orleans.fr
%
% Based on the instruction file for ACL 2013 
% which in turns was based on the instruction files for previous 
% ACL and EACL conferences

%% Based on the instruction file for EACL 2006 by Eneko Agirre and Sergi Balari
%% and that of ACL 2008 by Joakim Nivre and Noah Smith

\documentclass[11pt]{article}
\usepackage{eacl2014}
\usepackage{times}
\usepackage{url}
\usepackage{latexsym}
\usepackage[english]{babel}
\usepackage[latin1]{inputenc}
\usepackage[T1]{fontenc}

\special{papersize=210mm,297mm} % to avoid having to use "-t a4" with dvips 
%\setlength\titlebox{6.5cm}  % You can expand the title box if you really have to

\title{Shrdlite Project Report}

\author{Alexander Sjosten
  \\\And
  Asbjorn Hagalin \\
  \\\And
  Behrouz Talebi  \\
  \\\And
  Runar Kristinsson \\}
\date{}

\begin{document}
\maketitle
\begin{abstract}
Summary of the report. We should write this last.
\end{abstract}


\section{Introduction}
What is our main objective?


\section{Background}
Explain Shrdlue, what is it and why is it important?


\section{Interpretation}
The interpreter will, given a parse tree from the parser, create a list of goals that should be given to the planner. Each goal will be a list of subgoals (a subgoal will be refered to as PDDL from here on out). When interpreting the the given parse tree from the parser, the interpreter module will check its interpretations (i.e. the list of goals) to the given world and rule out the interpretations that are impossible in the current world. If no interpretations are valid in the current world, there will be an interpretation error and if the list of goals contain more than one element there exists an ambiguity and it will be the job of the ambiguity resolution module to resolve the ambiguity.

\subsection{Interpreting quantifiers}
There exists three quantifiers, namely "the", "any" and "all" and the design choices of how to interpret each of the quantifiers will vary depending on the context.

\subsubsection{The "the" quantifier}
When interpreting the quantifier "the", it will be interpreted as a specific object. Assume you have the sentence "move the white ball to the red box". This sentence makes sence if it is uttered in a world that contains exactly one white ball and one red box. If there exists two or more white balls and/or two or more red boxes, it will be a situation where the ambiguity resolution will have to be called. Assume the utterance given above and a world containing two white balls (call them a and b) and one red box (call it c). The interpreter will in this situation create two goals, one that says a should be on top of c and one that says b should be on top of c.

\subsubsection{The "any" quantifier}
If the "any" quantifier is used, it will be interpreted as whichever object that fits the description. If the sentence is "move any white ball to the red box", it will be interpreted as in the case where there are two or more white balls, select whichever of these and move it to the red box. At the moment, the interpreter will simply select the first goal that is valid. However, there are optimizations to be done. The interpreter could check the world and remove goals that are far more difficult to plan. Imagine a world where one white ball is free and another is blocked, it would be most likely be easier to plan if the free ball is selected.

\subsubsection{The "all" quantifier}
In situations where the "all" quantifier is used, it becomes more difficult for the interpreter. When generating the goals, the interpreter should not only select the objects in a way that could make it easier for the planner but it must also make sure that no ambiguos goals occurs if there is no need for it. Again, assume a sentence of the form "move all the white balls to a red box" and there are three white balls (call them a, b and c) and three red boxes (call them d, e and f). The interpreter should not give a result of the shape:
\begin{verbatim}
[[ontop a d, ontop b e, ontop c f]
,[ontop a d, ontop b f, ontop c e]
,[ontop a e, ontop b d, ontop c f]
,[ontop a e, ontop b f, ontop c d]
,[ontop a f, ontop b d, ontop c e]
,[ontop a f, ontop b e, ontop c d]
]
\end{verbatim}
because the ambiguity resolution module will see this as an ambiguity and try and resolve it. However, no ambiguity exists, so the interpreter must make sure that situations like this can't be seen as an ambiguity.

\section{Ambiguity resolution}
The ambiguity resolver receives a list of goals from the interpreter, and given that a list contains more than one element (goals), it will try to resolve the ambiguity. In case of a single goal from the interpreter we pass it forward to the planner. When an ambiguity is encountered, a clarification question is asked of the user.
When given a list of goals (each goal consists of subgoals, PDDLs) and a copy of the world (the current state of the stacks and objects), an ambiguity is encountered when there is more than one object that matches either the source or the destination.

\subsection{Types of ambiguity}
There will reside two types of ambiguity in the given context; either the object the user wants to move is ambiguous, in that it has more than one possible candidate (source) in the current world; or the user wants to move an object to a place (destination) that has more than one possible candidate in the given world. For example the utterance: "Move the white ball to the red box"
in a world with two white balls (or more) will be matched by the source case, and the same utterance will be matched by the destination case, in a world that contains two red boxes (or more).
In a world that contains multiple candidates for both source and destination, the ambiguity resolver asks the user to specify which source the user intends to move, and subsequently follows up with a second question regarding the destination.
Either type of ambiguity will be over shape, color, size and spatial relation.

\subsection{Finding ambiguity}
When given multiple goals, the ambiguity resolver looks for objects that match in destination but not in source and or vice versa. Consideration has to be taken in account for shape, size, color and spatial relation to make sure that there is an exact match in two or more objects given the goals. Every match is saved for later presentation to the user in the clarification question.

\subsection{Resolving ambiguity}
After the ambiguities has been found, the list of  objects that match in either destination and or source, together with corresponding source or destination is presented in a list to the user in a clarification question. Here the user is asked to make choices which in turn will be used for a new request to the program.
In case of multiple identical objects that matches (i.e. a world with only 10 small white boxes), it would be pointless to ask the user of which one one the user intends. Instead the user is told that there is multiple objects the user could be referring to and an arbitrary instance is picked to move.

\section{Planner}
From the beginning we wanted to create a planner that would make as few moves as possible to succeed at it's goal. We ended up creating Breadth first search, with heuristics that gives the algorithm lower bound and a specific node to go down. How does it work?

To find the lower bound we have a very simple cost function, it counts how many blocks are above the blocks that need to be moved to reach goal state. For example, if the goal is to put block a on top of block b, block a has 2 blocks on top of it and block b has 1 block on top of it. Then the lower bound would be (2+1) + 1 = 4. The reason for +1 is because we have to move a on top of b. 

To find the specific nodes to start at we sort the possible moves in the world, so we first move blocks in the stacks that are used in the goal, starting in the stack with more blocks above block involved in the goal. For example if we use the example above then we would  start with moves from the stack where block a is, because block a has more blocks on top of it, then we would move from the stack where block b is and then the rest of the moves

With these heuristics the algorithm uses Breadth first search, looking for goal state at the given lower bound first using most likely moves.

\section{Conclusion}
What have we learned, did we achieve the goals given in the introduction etc.

\section{Acknowledgments}
Probably not necessary for this report?

% If you use BibTeX with a bib file named eacl2014.bib, 
% you should add the following two lines:
\bibliographystyle{acl}
\bibliography{eacl2014}

\section*{Appendix}
What have each individual done.

\end{document}
